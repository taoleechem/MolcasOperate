

%----------------------------------------------------------------------------------------
%	PACKAGES AND DOCUMENT CONFIGURATIONS
%----------------------------------------------------------------------------------------

\documentclass{article}
\usepackage[margin=1in]{geometry}
\usepackage[version=3]{mhchem} % Package for chemical equation typesetting
\usepackage{siunitx} % Provides the \SI{}{} and \si{} command for typesetting SI units
\usepackage{graphicx} % Required for the inclusion of images
\usepackage{natbib} % Required to change bibliography style to APA
\usepackage{amsmath} % Required for some math elements 

\setlength\parindent{0pt} % Removes all indentation from paragraphs

\renewcommand{\labelenumi}{\alph{enumi}.} % Make numbering in the enumerate environment by letter rather than number (e.g. section 6)

%\usepackage{times} % Uncomment to use the Times New Roman font

%----------------------------------------------------------------------------------------
%	DOCUMENT INFORMATION
%----------------------------------------------------------------------------------------

\title{Lab Report of Laser Spectroscopy \\ Demonstration of 200kHz System} % Title

\author{Tao \textsc{Li}} % Author name

\date{\today} % Date for the report

\begin{document}

\maketitle % Insert the title, author and date


%----------------------------------------------------------------------------------------
%	SECTION 1
%----------------------------------------------------------------------------------------

\section{Questions}
1. What do we mean by intracavity frequency doubling in the pump laser?\par
It means that a non-linear crystal, which could generate doubling frequency under SHG, is placed within the laser cavity. And a dichroic mirror in the cavity could only permit doubling frequency out while has a high reflection index to fundamental frequency.\\
\par
2. What is the mode locking mechanism used in the Ti:sapphire laser?\par
Ti:sapphire crystal has Kerr effect, a non-linear effect of refractive index that $n_m=n_0+In_2$. When placed in the resonator, it has no(slight) focusing effect of weak light while having strong focusing effect of intense light. After travelling back and forth for many times, ultimately it would produce only a single, intense pulse. This is called \textit{passive mode locking}.   \\
\par 
3. Why do we need two prisms in the cavity?\par 
The two prisms form a prism conpressor system that to compansate the intra-cavity laser dispersion of Ti:sapphire laser. In detail, when laser beam goes through the first prism, different wavelengths would travel different paths. After the second prism, all wavelengths are adjusted to the same direction with different time, which would shorten the duration of laser pause.\\
\par 

4. What is the function of the adjustable slit in the Ti:sapphire laser?
\par 
Adjustment of slit width changes the bandwidth of laser and inhibits secondary laser peak. \\
\par 
5. What is a regenerative amplifier?
\par 
In the regenerative amplifier, the input seed pulse enters the amplifier cavity and the Ti:sapphire crystal and a short and intense pulse is generated. After multiple passes in the gain crystal, the amplified pulse exits the amplifier.\\
\par 
6. What type of Q-switch is used in the amplifier? What is the function of the Q-switcher?
\par 
Electro-optic Q-switch.\\
It requires a linear polarization filter and an EO polarization rotator in the resonator. After low intense laser being rotated of polarization, the beam cannot pass through the linear polarization filter. Unitil it develops to be a giant and intense laser, the beam outputs when the linear polarization filter is removed. In short, EO Q-switch is to generate high intense laser beam.\\

\par 
7. What method is used to compress the pulse in the system?
\par 
Mode locking with Kerr effect is used to compress the laser to short pulse and a pair of prisms is used to shorten the pulse duration by compensation of temporal dispersion.\\
 Consequencely, chirped pulse amplification (CPA) also compresses the dispersed pulse. \\
\par 
8. Why do we need to compress the pulses?
\par 
We cpmpress the pulse to a short duration pulse that helps to study fast chemical reaction and generate extremely high power.\\
\par 
9. What is the range of wavelength available for the OPA(Signal and Idler)? What is the wavelength of the idler if the signal is 580nm?
\par 
The wavelength availbale for signal and idler should obey the momentum consevation principle and  phase matching condition. In general, $n_p\upsilon_p=n_i\upsilon_i+n_s\upsilon_s$.\\
Suppose the pump beam has wavelength of 500nm, according to
\begin{equation*}
\frac{1}{\lambda_p}=\frac{1}{\lambda_i}+\frac{1}{\lambda_s}
\end{equation*}
we find the wavelength of idler is 4.67$\mu m$.


\end{document}
